% CREATED BY DAVID FRISK, 2014
Management of historical data in continuous integration systems\\
%A Subtitle that can be Very Much Longer if Necessary\\
ANDERS GUSTAFSSON\\
JONAS LERGELL\\
Department of Computer Science and Engineering\\
Chalmers University of Technology\\

\thispagestyle{plain}			% Supress header 
\section*{Abstract}
This master's thesis reports on the design and evaluation of two different prototypes for management of historical data in a continuous integration system at Ericsson. Continuous integration is an agile software practice where code is checked in frequently and subsequently built and tested automatically. Due the increasing maturity of this practice at Ericsson, the frequency of these builds is increasing and as such much more data about the development process is generated. The constructed prototypes where designed specifically to address problems related to scalability and performance of the current continuous integration systems in use at the company. Evaluation shows that the prototypes solves scalability problems and increases performance of current systems by separating live from stale data. The value of the prototypes is further motivated by investigating how historical data in a continuous integration system can be utilized for the benefit of the company.
% KEYWORDS (MAXIMUM 10 WORDS)
\vfill
Keywords: data management, continuous integration, relational databases, nosql, polyglot persistence, big data

\newpage				% Create empty back of side
\thispagestyle{empty}
\mbox{}