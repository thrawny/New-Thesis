% CREATED BY DAVID FRISK, 2014
\appendix
\chapter{Appendix 1}
\label{appendix1}
\section{Queries}
\hiddensubsection{Relational queries}'
\label{relationalqueries}
\hiddensubsubsection{Get build}
\label{q:getbuild}
Query in CIMS: Get build by id \\
SQL: {\tt SELECT * FROM cims\_build WHERE id = 'some\_id'} \\
Example input: {\tt CXS101289\_LLVDRAGON\_MKX\_SPRINT3\_R1A03\_140605084139 } \\
Example output: \\
This query is simply used to uniquely identify a build.

\hiddensubsubsection{Get build information}
\label{q:getbuildinfo}
Query in CIMS: Get build information by id \\
SQL: 
\begin{verbatim}
SELECT product_name, revision_name, verdict, start, stop
FROM cims_build WHERE id = 'some_id'
\end{verbatim}
Example input: {\tt CXS101289\_LLVDRAGON\_MKX\_SPRINT3\_R1A03\_140605084139 }  \\
Example output: \\
Given a unique build id, return the basic information for that build, i.e. product, revision, verdict, start/stop times.

\hiddensubsubsection{Get build for root test suite}
\label{q:getbuildforroot}
Query in CIMS: Get build by job event simple id \\
SQL: 
\begin{verbatim}
SELECT cims_build.* FROM cims_build
INNER JOIN cims_build_job_mapping
ON cims_build_job_mapping.build_id = cims_build.id   
WHERE cims_build_job_mapping.job_id = 'some_simple_id'
\end{verbatim} 
Example input: {\tt 1234567 }  \\
Example output: \\
This query should retrieve the build that a root test suite belongs to.

%\hiddensubsubsection{Get trouble reports for build}
%\label{q:getrforbuild}
%Query in CIMS: Get deliverables for build \\
%Example input: {\tt CXS101289\_LLVDRAGON\_MKX\_SPRINT3\_R1A03\_140605084139 }  \\
%Example output: \\
%Given a unique build id, return a list of deliverables valid for that build. In domain terms this means %retrieving all trouble reports that affect the build.

%\hiddensubsubsection{Get trouble report fixes for build}
%\label{q:getrfixforbuild}
%Query in CIMS: Get deliverables for build \\
%Example input: {\tt CXS101289\_LLVDRAGON\_MKX\_SPRINT3\_R1A03\_140605084139 }  \\
%Example output: \\
%Given a unique build id, retrieve all trouble report fixes that the build delivered.

%\hiddensubsubsection{Get trouble reports for product and revision}
%\label{q:gettrforprod}
%Query in CIMS: Get deliverables for product and revision \\%%Example input: {\tt (Product=ndpgsn\_5\_0\_llvdragon\_mkx\_sprint3,\ Revision=R1A03) }  \\
%Example output: \\
%Given a combination of a product name and a revision name, return a list of deliverables that affect that combination. In domain terms this means returning trouble reports that affect the combination.

%\hiddensubsubsection{Get test suite children}
%\label{q:gettestsuitechildren}
%Query in CIMS: Get job event children \\
%SQL: {\tt SELECT * FROM cims\_job\_event WHERE parent\_jobevent\_id = 'some\_id' }
%Example input: {\tt fff70f3a-c69e-44c1-b9f5-7e5ae5a59610 } \\
%Example output: \\
%For implementational reasons, both test cases and test suites are represented as job events in the relational data model of CIMS. As such, a job event id uniquely identifies a specific test case or test suite in a specific build. This query should, given a job event id, return its children. In domain terms this means that for a specific build, return the child nodes of a test suite. These child nodes can be test cases (leaf nodes) or test suites (tree nodes).

%\hiddensubsubsection{Get test case in build}
%\label{q:gettcinbuild}
%Query in CIMS: Get job event by id \\
%SQL: {\tt SELECT * FROM cims\_job\_event WHERE id = 'some\_id' } \\
%Example input: {\tt fff70f3a-c69e-44c1-b9f5-7e5ae5a59610 } \\%%Example output: \\
%This query should retrieve information about a test case in a specific build. In the relational data model of CIMS this can be translated to retrieving a job event by its unique id.

%\hiddensubsubsection{Get test suite in build}
%\label{q:gettcinbuild}
%Query in CIMS: Get job event by id \\
%SQL: {\tt SELECT * FROM cims\_job\_event WHERE id = 'some\_id' } \\
%Example input: {\tt fff70f3a-c69e-44c1-b9f5-7e5ae5a59610 } \\
%Example output: \\
%This query should retrieve information about a test suite in a specific build. In CIMS, test suites and test cases are both represented by the job event entity and as such the query in CIMS is the same as the previous query.

\hiddensubsubsection{Get root test suite for test case/test suite}
\label{q:getrootts}
Query in CIMS: Get root job event \\
SQL:
\begin{verbatim}
SELECT cims_job_event.* FROM cims_job
INNER JOIN cims_job_event 
ON cims_job_event.id = cims_job.root_jobevent_id
WHERE cims_job.job_name =
(SELECT job_name FROM cims_job_event WHERE id = 'some_id')

\end{verbatim}
Example input: {\tt fff70f3a-c69e-44c1-b9f5-7e5ae5a59610 } \\
Example output: \\
Given the unique id of a test case or test suite, this query should retrieve the related root test suite.

\hiddensubsubsection{Get test case by name}
\label{q:gettcbyname}
Query in CIMS: Get job event by name \\
SQL: {\tt SELECT * FROM cims\_job\_event WHERE name = 'some\_name' } \\
Example input: {\tt tc\_48\_027\_handover\_roaming\_restricted } \\
Example output: \\
Every test case has a descriptive name. This query should retrieve all test cases with a given name. This means retrieving test cases from different builds, since builds often share the same test suites and test cases.

%\hiddensubsubsection{Get test suite for test case}
%\label{q:gettsfortc}
%Query in CIMS: Get parent job event \\
%SQL:
%\begin{verbatim}
%SELECT * FROM cims_job_event
%WHERE cims_job_event.id =
%(SELECT parent_jobevent_id FROM cims_job_event WHERE id = 'some_id')
%\end{verbatim}
%Example input: {\tt fff70f3a-c69e-44c1-b9f5-7e5ae5a59610 } \\
%Example output: \\
%For a test case in a specific build, retrieve the test suite it belongs to in that build.

\hiddensubsection{Elasticsearch queries}
\hiddensubsubsection{Get build}
\label{q:getbuildEs}
Query for Elasticsearch: \\
\begin{verbatim}
POST http://localhost:18843/build/_search
{ 
    "query" : {
        "ids" : { 
             "values" : ["CXS101289_10_LLVCORA"]
        }
    }
}
\end{verbatim}

%Query for MongoDB:
%\begin{verbatim}
%db.build.findOne( { "_id": "some_id" } )
%\end{verbatim}
%This query is simply used to uniquely identify a build.


\hiddensubsubsection{Get build information}
\label{q:getbuildEs}
Query for Elasticsearch:
\begin{verbatim}
POST http://localhost:18843/build/_search
{ 
    "query" : {
        "ids" : { 
            "values" : ["CXS101289_10_LLVCORA"]
        }
    },
    "fields" :  ["product_name","product_revision",
                    "verdict","start","end"]
}
\end{verbatim}

%Query for MongoDB:
%\begin{verbatim}
%db.build.findOne( 
%    { "_id": "some_id" }, 
%    { "product_name": 1, "product_revision": 1, 
%      "verdict": 1, "start": 1, "end": 1 } 
%)
%\end{verbatim}

\hiddensubsubsection{Get build for root test suite}
Query for Elasticsearch:\\
This query requires two round trips to Elasticsearch, first to get the id of the job's build, latter to get nformation about that build.\\
\begin{verbatim}
POST http://localhost:18843/job/_search
{ 
    "query" : {
        "ids" : { 
            "values" : ["1234"]
        }
    },
    "fields" :  ["build_job_mapping_build_id"]
}
\end{verbatim}
The result from this query is, which is an build id is then used as parameter to the "Get build information" query described in \ref{q:getbuildinfoEs}. 


%Query for MongoDB:
%\begin{verbatim}
%build_id = db.job.findOne( 
%    { "job_name": "some_name" }, 
%    { "build_job_mapping_build_id": 1 } 
%)
%db.build.findOne( { "_id": build_id } )
%\end{verbatim}

\hiddensubsubsection{Get root test suite for test case/test suite}
\label{q:getroottsEs}
Query for Elasticsearch:\\
This query requires two round trips to Elasticsearch, first to get the job event's job name, this name is then used o retrieve information about that job.
\begin{verbatim}
POST http://localhost:18843/job_event/_search
{ 
    "query" : {
        "ids" : { 
            "values" : ["fff70f3a-c69e-44c1-b9f5-7e5ae5a59610"]
        }
    },
    "fields" :  ["job_name"]
}

POST http://localhost:18843/job/_search
{ 
    "query" : {
        "ids" : { 
            "values" : ["result from previous query"]
        }
    }
}
\end{verbatim}

%Query for MongoDB:
%\begin{verbatim}
%job_name = db.job_event.findOne( 
%    { "_id": "some_id" }, 
%    { "job_name": 1 } 
%)
%db.job.findOne( { "_id": job_name } )
%\end{verbatim}

\hiddensubsubsection{Get test case by name}
Query for Elasticsearch:
\label{q:gettcbynameEs}
\begin{verbatim}
POST http://localhost:18843/job_event/_search
{
    "query" : {
        "filtered" : {
            "filter" : {
                "term" : {
                            "name" : "some_name"
                }
            }
        }
    }
}
\end{verbatim}

%Query for MongoDB:
%\begin{verbatim}
%db.job_event.find( { "name": "some_name" } )
%\end{verbatim}